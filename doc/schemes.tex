\documentclass{article}
\usepackage[utf8]{inputenc}
\usepackage{amsmath}
\usepackage{tikz}
\usepackage{xcolor}
\usepackage{amssymb}
\usepackage{graphicx}
\usepackage{mathtools}
\usetikzlibrary{quotes,angles, decorations.pathreplacing}

\title{Trashcan}
\author{adrien.laydu }
\date{March 2022}

\begin{document}

\def\nn{8}
%I honestly want to murder myself after seeing this code
\begin{figure}
    \centering
    \resizebox{\columnwidth}{!}{%
            \begin{tikzpicture}
                \draw[step=1, thin] (0,0) grid (\nn, \nn);
                \draw[->, thick] (-0.9,\nn/2) -- ( \nn + 0.9, \nn/2);
                \draw[->, thick] (\nn/2, -0.9) -- (\nn/2, \nn + 0.9);
                \node[rectangle,draw=black,fill=black!10!white, minimum width = 1cm, minimum height = 1cm] (r) at (2.5,5.5){$\omega_0$};
                \node[rectangle,draw=black,fill=black!10!white, minimum width = 1cm, minimum height = 1cm] (r) at (8-2.5,8-5.5){$\omega_0^*$};
                \node[rectangle,draw=black,fill=black!20!white, minimum width = 1cm, minimum height = 1cm] (r) at (3.5,5.5){$\omega_1$};
                \node[rectangle,draw=black,fill=black!20!white, minimum width = 1cm, minimum height = 1cm] (r) at (8-3.5,8-5.5){$\omega_1^*$};
                \node[rectangle,draw=black,fill=black!20!white, minimum width = 1cm, minimum height = 1cm] (r) at (4.5,5.5){$\omega_2$};
                \node[rectangle,draw=black,fill=black!20!white, minimum width = 1cm, minimum height = 1cm] (r) at (8-4.5,8-5.5){$\omega_2^*$};
                \node[rectangle,draw=black,fill=black!10!white, minimum width = 1cm, minimum height = 1cm] (r) at (5.5,5.5){$\omega_3$};
                \node[rectangle,draw=black,fill=black!10!white, minimum width = 1cm, minimum height = 1cm] (r) at (8-5.5,8-5.5){$\omega_3^*$};
                \node[rectangle,draw=black,fill=black!20!white, minimum width = 1cm, minimum height = 1cm] (r) at (2.5,4.5){$\omega_4$};
                \node[rectangle,draw=black,fill=black!20!white, minimum width = 1cm, minimum height = 1cm] (r) at (8-2.5,8-4.5){$\omega_4^*$};
                \node[rectangle,draw=black,fill=black!40!white, minimum width = 1cm, minimum height = 1cm] (r) at (3.5,4.5){$\omega_5$};
                \node[rectangle,draw=black,fill=black!40!white, minimum width = 1cm, minimum height = 1cm] (r) at (8-3.5,8-4.5){$\omega_5^*$};
                \node[rectangle,draw=black,fill=black!40!white, minimum width = 1cm, minimum height = 1cm] (r) at (4.5,4.5){$\omega_6$};
                \node[rectangle,draw=black,fill=black!40!white, minimum width = 1cm, minimum height = 1cm] (r) at (8-4.5,8-4.5){$\omega_6^*$};
                \node[rectangle,draw=black,fill=black!20!white, minimum width = 1cm, minimum height = 1cm] (r) at (5.5,4.5){$\omega_7$};
                \node[rectangle,draw=black,fill=black!20!white, minimum width = 1cm, minimum height = 1cm] (r) at (8-5.5,8-4.5){$\omega_7^*$};
            \end{tikzpicture}%
        }
    \caption{Caption}
    \label{fig:my_label}
\end{figure}
\newpage
\def\scl{3}
\def\points{(\scl * -1, \scl * 1) / 0110, (\scl * -0.5, \scl * 1) / 1011, (\scl * 0.5, \scl * 1) / 1110, (\scl * 1, \scl * 1) / 0011,
(\scl * -1, \scl * 0.5) / 1101, (\scl * -0.5, \scl * 0.5) / 0000, (\scl * 0.5, \scl * 0.5) / 0101, (\scl * 1, \scl * 0.5) / 1000,
(\scl * -1, \scl * -0.5) / 0111, (\scl * -0.5, \scl * -0.5) / 1010, (\scl * 0.5, \scl * -0.5) / 1111, (\scl * 1, \scl * -0.5) / 0010,
(\scl * -1, \scl * -1) / 1100, (\scl * -0.5, \scl * -1) / 0001, (\scl * 0.5, \scl * -1) / 0100, (\scl * 1, \scl * -1) / 1001,
}
\begin{figure}
    \centering
    \begin{tikzpicture}
        \foreach \Point/\PointLabel in \points{
            \draw[fill=black] \Point circle (0.05) node[above right] {$\PointLabel$};
        }
        \coordinate (o) at (0,0);
        \coordinate (e) at (0.5 * \scl,\scl);
        \coordinate (x) at (1,0);
        \draw[->, thick] (-1.4 * \scl,0) -- (1.4 * \scl, 0) node[above right] {$\mathcal R(\omega)$};
        \draw[->, thick] (0, -1.4 * \scl) -- (0, 1.4 * \scl) node[above right] {$\mathcal I(\omega)$};
        \draw[->, red, ultra thick] (0,0) -- (0.5 * \scl ,\scl) node [midway, above left] {$r$};
        \draw pic ["$\theta$", draw=red, text = red, -, angle eccentricity=1.2, angle radius=\scl * 0.25 cm] {angle = x--o--e};    
    \end{tikzpicture}
    \caption{Encoding 4-bit words into frequencies.}
    \label{fig:encoding}
\end{figure}

\newpage
 \def \n{6}
 \def \scl{0.5}
 %Pain
 \def \rects {
    (0, -\scl)/red/0,(\scl, 0)/blue/1,(0, -2*\scl)/green!70!black/2,(\scl,  -\scl)/yellow/3,(2* \scl,  0)/orange/4,(\scl,  \scl + 0)/purple/5,(\scl,  2*\scl + 0)/brown/6,(2*\scl,  \scl+ 0)/gray/7,(3 * \scl,  0)/pink/8,(2*\scl, -\scl)/blue!40!white/9, (\scl, -2*\scl)/blue!40!white/9, (0, -3 *\scl)/pink/8,(0, -4 * \scl) /gray/7,(\scl,  -3 * \scl) /brown/6, (2*\scl,  -2*\scl)/purple/5,(3*\scl,  - \scl)/orange/4,(4 * \scl,  0)/yellow/3, (3 * \scl,  \scl)/green!70!black/2,(2 * \scl, 2 * \scl)/blue/1,(\scl,  3 * \scl + 0)/red/0,(0,0)/white!80!black/r
    %(0, 0)/green!70!black!30!black/r
 }
\begin{tikzpicture}
    %\draw[step= \scl, thin] (-\n * \scl,-\n * \scl) grid (\n * \scl, \n * \scl);
    \draw[->, thick] (-\n * \scl - 0.5,0) -- (\n * \scl + 0.5, 0) node[above right] {$\mathcal R(\omega)$};
    \draw[->, thick] (0, -\n * \scl - 0.5) -- (0, \n * \scl + 0.5) node[above right] {$\mathcal I(\omega)$};
    \foreach \Point/\Col/\Lab in \rects{
        \node[rectangle,fill=\Col,  minimum width = \scl cm, minimum height = \scl cm] (r) at \Point {$\Lab$};
    }
\end{tikzpicture}
$$\mathcal B_m = I(f_m > 0.5), \hspace{3pt} m \in \{1,2,3\}$$
$$\mathcal P (\{\mathcal B_1, \mathcal B_2, \mathcal B_3\}) \rightarrow \{Color\#1, Color\#2, \dotsc, Color\#2^m\}$$
\begin{align*}
	\{Color\#1, Color\#2, \dotsc, Color\#2^m\} = \{&f_1 \leq 0.5 \hspace{1pt} \wedge \hspace{1pt} f_2 \leq 0.5 \hspace{1pt} \wedge \hspace{1pt} f_3 \leq 0.5,\\
	&f_1 \leq 0.5 \hspace{1pt} \wedge \hspace{1pt} f_2 \leq 0.5 \hspace{1pt} \wedge \hspace{1pt} f_3 > 0.5,\\
	&\dots,\\
	& f_1 > 0.5 \hspace{1pt} \wedge \hspace{1pt} f_2 > 0.5 \hspace{1pt} \wedge \hspace{1pt} f_3 > 0.5
\end{align*}


\makeatletter
\newcommand*\dashline{\rotatebox[origin=c]{90}{$\dabar@\dabar@\dabar@$}}
\makeatother
$$h = \underbrace{\textcolor{red}{\underline{1}100}}_{f_1} \dashline \underbrace{\textcolor{red}{1000}}_{f_2} \dashline \underbrace{\textcolor{red}{0110}}_{f_3} \dashline \textcolor{blue}{\underline{1}110 \dashline 0101 \dashline 1011 \dashline }\textcolor{green!70!black}{\underline{0}100 \dashline 1111 \dashline 0110 \dashline }\textcolor{orange}{\underline{0}010 \dashline 1001 \dashline 0110}\dots\textcolor{blue!40!white}{\underline{0}010 \dashline 1001 \dashline 0101 \dashline }\textcolor{white!40!black}{1001 \dashline 0100}$$

\vspace{8pt}
\begin{equation*}
	p_i = \begin{cases} 
	    \textcolor{red}{b_{0,i}} \oplus \textcolor{blue}{b_{1,i}} \oplus \textcolor{green!70!black}{b_{2,i}} \oplus \dots \oplus \textcolor{blue!40!white}{b_{9,i}} = \bigoplus_{k=0}^9 b_{k,i} & 0 \leq i < 12\\
	    \textcolor{white!40!black}{h_{120 + i - 12} \oplus h_{120 + i - 8}} & 12 \leq i < 16
	    \end{cases}
\end{equation*}


\vspace {8pt}

\def\palette{
    0/red/11111,1/blue/11110,2/green/11101,3/gray/11100,4/white/\vdots,5/orange/00011,6/yellow/00010,7/brown/00001,8/violet/00000}

%First palette drawing
\def\firstthird{\scl*4*2.5}
\begin{tikzpicture}
 \node[rectangle, draw=black, minimum width = \scl*6 cm, minimum height = \scl*2.5 cm] (r) at (0, \scl * 10.5cm){$p_0 := \bigoplus\limits_{k=0}^9 b_{12k}$};
 \node[rectangle, draw=black, minimum width = \scl*6 cm, minimum height = \scl*2.5 cm] (r) at (0, \scl * 8cm){$p_1 := \bigoplus\limits_{k=0}^9 b_{12k + 1}$};
 \node[rectangle, draw=black, minimum width = \scl*6 cm, minimum height = \scl*2 cm] (r) at (0, \scl * 5.75cm){$\vdots$};
 \node[rectangle, draw=black, minimum width = \scl*6 cm, minimum height = \scl*2.5 cm] (r) at (0, \scl * 3.5cm){$p_{14} := b_{122} \oplus b_{126} $};
 \node[rectangle, draw=black, minimum width = \scl*6 cm, minimum height = \scl*2.5 cm] (r) at (0, \scl * 1cm){$p_{15} := b_{123}\oplus b_{127}$};

\draw[thick]  (\scl*3cm, 10.5*\scl) --  (\scl*5, 10.5*\scl);
\draw[thick]  (\scl*3cm, 8*\scl) --  (\scl*5, 8*\scl);
\draw[thick]  (\scl*5cm, 10.5*\scl) --  (\scl*5, 9.75*\scl);
\draw[thick]  (\scl*5cm, 8*\scl) --  (\scl*5, 9.25*\scl);
\draw[thick,->]  (\scl*5cm, 9.75*\scl) --  (\firstthird-2.25*\scl, 9.75*\scl);
\draw[thick,->]  (\scl*5cm, 9.25*\scl) --  (\firstthird-2.25*\scl, 9.25*\scl);

\draw[thick]  (\scl*3cm, 3.5*\scl) --  (\scl*5, 3.5*\scl);
\draw[thick]  (\scl*3cm, 1*\scl) --  (\scl*5, 1*\scl);
\draw[thick]  (\scl*5cm, 3.5*\scl) --  (\scl*5, 2.25*\scl);
\draw[thick]  (\scl*5cm, 1*\scl) --  (\scl*5, 1.75*\scl);
\draw[thick,->]  (\scl*5cm, 1.75*\scl) --  (\firstthird-2.25*\scl, 1.75*\scl);
\draw[thick,->]  (\scl*5cm, 2.25*\scl) --  (\firstthird-2.25*\scl, 2.25*\scl);

 \node[rectangle, draw=black, minimum width = \scl*4.5 cm, minimum height = \scl*2.5 cm] (r) at (\firstthird, \scl * 9.5cm){$c_0 := p_0||p_1$};
 \node[rectangle, draw=black, minimum width = \scl*4.5 cm, minimum height = \scl*2.5 cm] (r) at (\firstthird, \scl * 2cm){$c_7 := p_{14}||p_{15}$};
 \node[rectangle, draw=black, minimum width = \scl*4.5 cm, minimum height = \scl*5 cm] (r) at (\firstthird, \scl * 5.75cm){$\vdots$};


\draw[thick] (\firstthird + 2.25*\scl, \scl * 9.5cm) -- (\firstthird + 4.25*\scl, \scl * 9.5cm);
\draw[thick] (\firstthird + 4.25*\scl, \scl * 9.5cm) -- (\firstthird + 4.25*\scl, \scl * 12cm);
\draw[thick,->] (\firstthird + 4.25*\scl, \scl * 12cm) -- (7*2.5*\scl, \scl * 12cm);

\draw[thick] (\firstthird + 2.25*\scl, \scl * 2cm) -- (\firstthird + 4.25*\scl, \scl * 2cm);
\draw[thick] (\firstthird + 4.25*\scl, \scl * 2cm) -- (\firstthird + 4.25*\scl, \scl * 3cm);
\draw[thick,->] (\firstthird + 4.25*\scl, \scl * 3cm) -- (7*2.5*\scl, \scl * 3cm);

\foreach \Y/\Col/\Lab in \palette{
        \node[rectangle,fill=\Col,  minimum width = \scl*5 cm, minimum height = \scl*1.5 cm] (r) at (\scl*8*2.5, \scl * \Y * 1.5) {\texttt{\Lab}};
    }
    \draw [decorate,
    decoration = {brace, mirror}, ultra thick] (\scl*9*2.5 + 0.25*\scl,-0.75*\scl) --  (\scl*9*2.5+ 0.25*\scl,8*1.5*\scl + 0.75*\scl) node[pos=0.5,right=10pt,black]{32 colors};
\end{tikzpicture}
\vspace{8pt}

%Palette shift drawing
\def\paletteshifted{
    0/green/11111,1/gray/11110,2/green!60!black/11101,3/pink/11100,4/white/\vdots,5/brown/00011,6/violet/00010,7/red/00001,8/blue/00000}
\def\maxy{8*1.5*\scl + 0.75*\scl}
\begin{tikzpicture}
    \foreach \Y/\Col/\Lab in \palette{
        \node[rectangle,fill=\Col,  minimum width = \scl*5 cm, minimum height = \scl*1.5 cm] (r) at (0, \scl * \Y * 1.5) {\texttt{\Lab}};
    }
    \draw [decorate,
    decoration = {brace}, ultra thick] (-2.7*\scl,-0.75*\scl) --  (-2.7*\scl,8*1.5*\scl + 0.75*\scl) node[pos=0.5,left=10pt,black]{32 colors};
    \draw[->, thick] (\scl * 5 * 1.25, \maxy) -- (\scl * 5 * 1.25, 0) node[midway, right]{$h \mod 29$ };
    \foreach \Y/\Col/\Lab in \paletteshifted{
        \node[rectangle,fill=\Col,  minimum width = \scl*5 cm, minimum height = \scl*1.5 cm] (r) at (\scl*5*2.5, \scl * \Y * 1.5) {\texttt{\Lab}};
    }
    \node[text width = \scl*7cm] at (\scl * 19cm, \scl * 6){if $w(h) \mod 2 = 1$, reverse order};
\end{tikzpicture}

%$$\sum\text{index} = \sum\limits_{k=0}^{128}  I(b_k = 1)k \mod 31$$ 

\vspace{8pt}
Proof that $\Tilde{x} \neq x \mod 23$, where $\Tilde{x}$ is $x$ with 2 bits of the same group changed:
Let $\ell$ and $\ell+12m$ the indices of the 2 flipped bits, with $0 < m \leq 10$.

Ring of $\mathbb Z_p$ is an integral domain.

\begin{itemize}
    \item Case $b_\ell = 1, b_{\ell + 12m} = 0$:
    
    \begin{align*}
        & x = \Tilde{x} = x - 2^\ell + 2^{\ell + 12m} \mod 23\\ 
        \implies & x = x + 2^\ell\left(2^{12m} - 1\right) \mod 23\\
        \implies & 0 = 2^\ell\left(2^{12m} - 1\right) \mod 23\\
        \implies & 0 = 2^{12m} - 1 \mod 23\\
        \implies & 1 = 2^{12m}  \mod 23
    \end{align*}
    which has no solution for $1 < m \leq 10$.
    
    \item Case $b_\ell = 0, b_{\ell + 12m} = 1$:
    
    \begin{align*}
        & x = \Tilde{x} = x - 2^\ell + 2^{\ell + 12m} \mod 23\\ 
        \implies & x =  x + 2^\ell\left(1 - 2^{12m}\right) \mod 23\\
        \implies & 0 = 2^\ell\left(1 - 2^{12m}\right) \mod 23\\
    \end{align*}
    which has no solution for $1 < m \leq 10$, same as the previous case.
    
    \item Case $b_\ell = 1, b_{\ell + 12m} = 1$:
        \begin{align*}
        & x = \Tilde{x} = x + 2^\ell + 2^{\ell + 12m} \mod 23 \\ 
        \implies & x = x + 2^\ell\left(1 + 2^{12m}\right) \mod 23\\
        \implies & 0 = 2^\ell\left(1 + 2^{12m}\right) \mod 23\\
        \implies & 0 = 1 + 2^{12m} \mod 23\\
        \implies & -1 = 2^{12m}  \mod 23
    \end{align*}
    which has no solution for $1 < m \leq 10$.
    
    \item Case $b_\ell = 0, b_{\ell + 12m} = 0$:
        \begin{align*}
        & x = \Tilde{x} = x - 2^\ell - 2^{\ell + 12m} \mod 23 \\ 
        \implies & x = x + 2^\ell\left(-1 - 2^{12m}\right) \mod 23\\
        \implies & 0 = 2^\ell\left(-1 - 2^{12m}\right) \mod 23\\
    \end{align*}
    which has no solution for $1 < m \leq 10$, same as the previous case
\end{itemize}

Attacks if shift is $\mod 23$:
\begin{itemize}
    \item Adv flips $n$ bits of the same index, keeping same modulo 23 :
    \begin{itemize}
        \item Effect : the same palette is used. If $n$ is even and $n > 2$, then all colors are exactly the same as intra group parity is the same
        \item "Impossible" for odd $n$ as palette is flipped
        \item Rare (I guess) for $n = 4$: maximum 8 if the ten bits are e.g. 1101011100
    \end{itemize}
    \item Adv flips $n$ bits, keeping same modulo 23 :
    \begin{itemize}
        \item Effect : the same palette is used. The colors corresponding to the $n$ indices are changed
        \item "Impossible" for odd $n$ as palette is flipped
    \end{itemize}
\end{itemize}

\vspace{8pt}
\begin{align*}
	x + 2^k - 2^\ell &= x \mod p\\
	2^k - 2^\ell &= 0 \mod p\\
	2^\ell \left(2^{k - \ell} - 1 \right) &= 0 \mod p\\
	2^{k-\ell} &= 1 \mod p\\
	k - \ell &= m \cdot |2|\mod p, \hspace{3pt} m \in \mathbb Z
\end{align*}
Problem : 11 is $-1 \mod 12 \implies$ flipping $b_i$ and $b_{i+11}$ might only change 1 color.

Attacks if shift is $\sum index$:

Lots (sum of 4 indices that are for the same function divide 31)

%>>> [(2 ** (12*m)) % 23 for m in range(11)]
%[1, 2, 4, 8, 16, 9, 18, 13, 3, 6, 12]

Properties:
\begin{itemize}
	\item Color choices : $\{\left(b_0  \oplus b_{12} \oplus \dots \oplus b_{108}\right), \left(b_1 \oplus b_{13} \oplus \dots \oplus b_{109}\right), \cdots, \left(b_{11} \oplus \dots \oplus b_{119}\right)\}$
	\item Palette shift : $h \mod 23$
	\item Invert palette direction : $w(h) \mod 2$
	\item Symmetry mode : $\sum\limits_{i : h_i = 1}i + 11 \mod 13$
\end{itemize}

Keeping same parity bits :
$$b^\prime_0 \oplus b^\prime_{12} \oplus \dots \oplus b^\prime _{108} = b_0  \oplus b_{12} \oplus \dots \oplus b_{108}$$

$$\bigoplus\limits_{i=0}^{9} b^\prime_{12i} = \bigoplus\limits_{i=0}^{10} b_{12i}$$
Implications :
\begin{itemize}
	\item If $b^\prime_i \neq b_i$ for an odd number of $i$, then the parity of the weight of $h$ changes and the palette is flipped.
	\item To keep the same palette shift, we must have $h = h^\prime \mod 23$, with $h^\prime = h + \sum\limits_k 2^k - \sum\limits_\ell 2^\ell$ for all $k : b_k = 0 \wedge b^\prime_k = 1$ and $\ell : b_\ell = 1 \wedge b^\prime_\ell = 0$. Because of the previous point, we must have $k + \ell = 0 \mod 2$.
\end{itemize}

\begin{align*}
	 h + \sum\limits_k 2^k - \sum\limits_\ell 2^\ell &= h \mod 23\\
	\sum\limits_k 2^k - \sum\limits_\ell 2^\ell &= 0 \mod 23
\end{align*}
$|2| \mod 23 = 11 \implies 2^{12i} \mod 23 = \left(2^{12}\right)^i \mod 23= \left( 2 \cdot 2^{11} \right)^i \mod 23= 2^i \mod 23$.

$2^{12i} \mod 23 = 2^i \mod 23$ for $i \in \{0, \cdots, 9\} = \{1, 2, 4, 8, 16, 9, 18, 13, 3, 6\} := M$

For $n = 2,4,6,8,10$ : find $n$ distinct elements $m_i \in M$ and a vector $\alpha \in \{-1, 1\}^n$ such that $\sum\limits_{i = 0}^9 \alpha_i m_i = 0 \mod 23$

\begin{itemize}
	\item $n=2$: It is impossible to find $m_1 \neq m_2$ such that $m_1 \pm m_2 = 0 \mod 23$.
	\item $n=4$: With Python, we found there are 84 possible choices for $m_0, m_1, m_2, m_3$ such that we can find a fitting $\alpha$. For example, $m_1 = 2, m_2 = 4, m_3 = 16, m_4 = 18$, we find $\alpha = \{1, -1, -1,1\}$.
	\item $n=6$: We found there are 280 possible choices for $m_0$ to $m_5$ such that we can find a fitting $\alpha$.
	\item $n=8$: We found there are 255 possible choices for $m_0$ to $m_7$ such that we can find a fitting $\alpha$.
	\item $n=10$: Picking $m_0$ to $m_9$ as every element of $M$, we can find (for example) $\alpha = \{1, -1, -1, 1,-1,1,-1,1,1,1\}$ that is fitting.
	\item In total, 620 "collisions" with same parity and same palette shift
\end{itemize}
Introducing the symmetries: in order to keep the same symmetry, we must have $\sum\limits_{i : h_i =1}i + 11 \mod 13 = \sum\limits_{i : h^\prime_i =1}i + 11 \mod 13$. That means a collision must have $\sum 12m_i\alpha_i = 0\mod 13$
\begin{itemize}
	\item $n=4$: The number of collision drops to 53 (IN TOTAL):.
	\item $n=6$: We found there are 292 (IN TOTAL) possible collisions.
	\item $n=8$: We found there are 230 (IN TOTAL) possible collisions.
	\item $n=10$:We found there are 25 (IN TOTAL) possible collisions).
	\item In total, 620 "collisions" with same parity and same palette shift
	\item However, each requires a specific combination of bits to be possible. Some of them are mutually exclusive. $\rightarrow$ divide by $\approx$ 16, 32, 64 $\rightarrow$  
\end{itemize}

If the bits are not from the same parity:

If two flipped bits of same value:

Same shift : 
\begin{align*}
	h + 2^k + 2^\ell &= h \mod p \\
	2^k + 2\ell &= 0 \mod p\\
	2^\ell( 2^{k-\ell} + 1) &= 0 \mod p\\
 	2^{k-\ell} &= -1 \mod p\\
\end{align*}
Which is not possible as long as we pick $p \neq 3$ and $p \neq 17$

Same shift : 
\begin{align*}
	h + 2^k - 2^\ell &= h \mod p \\
	2^k - 2\ell &= 0 \mod p\\
	2^\ell( 2^{k-\ell} - 1) &= 0 \mod p\\
 	2^{k-\ell} &= 1 \mod p\\
	k - \ell &= 0 \mod (|2| \mod p)\\
\end{align*}

Same symmetry :
$$k - \ell = 0 \mod (|2| \mod q)$$

Having both yield

$$k - \ell = 0 \mod \text{lcm}(|2| \mod p,|2| \mod q)$$

If we pick $p = 29, q = 23$ we have $(|2| \mod p,|2| \mod q) = (28, 11)$ and $\text{lcm}(28,11) = 308 > 256$

\newpage
\def \s{0.5cm}
\begin{tikzpicture}
	\node[rectangle] (r) at (\s * 10, 0){\texttt{h = 101011101010101\dots1010011110111}};
\end{tikzpicture}




























%Space for compilation window
\end{document}

\begin{itemize}
	\item $n=2$: It is impossible to find $m_1 \neq m_2$ such that $m_1 \pm m_2 = 0 \mod 23$.
	\item $n=4$: With Python, we found there are 84 possible choices for $m_0, m_1, m_2, m_3$ such that we can find a fitting $\alpha$. For example, $m_1 = 2, m_2 = 4, m_3 = 16, m_4 = 18$, we find $\alpha = \{1, -1, -1,1\}$.
	\item $n=6$: We found there are 280 possible choices for $m_0$ to $m_5$ such that we can find a fitting $\alpha$.
	\item $n=8$: We found there are 255 possible choices for $m_0$ to $m_7$ such that we can find a fitting $\alpha$.
	\item $n=10$: Picking $m_0$ to $m_9$ as every element of $M$, we can find (for example) $\alpha = \{1, -1, -1, 1,-1,1,-1,1,1,1\}$ that is fitting.
	\item In total, 620 "collisions" with same parity and same palette shift
\end{itemize}
Introducing the symmetries: in order to keep the same symmetry, we must have $\sum\limits_{i : h_i =1}i + 11 \mod 13 = \sum\limits_{i : h^\prime_i =1}i + 11 \mod 13$. That means a collision must have $\sum 12m_i\alpha_i = 0\mod 13$
\begin{itemize}
	\item $n=4$: The number of collision drops to 53 (IN TOTAL):.
	\item $n=6$: We found there are 292 (IN TOTAL) possible collisions.
	\item $n=8$: We found there are 230 (IN TOTAL) possible collisions.
	\item $n=10$:We found there are 25 (IN TOTAL) possible collisions).
	\item In total, 620 "collisions" with same parity and same palette shift
	\item However, each requires a specific combination of bits to be possible. Some of them are mutually exclusive. $\rightarrow$ divide by $\approx$ 16, 32, 64 $\rightarrow$  
\end{itemize}


















\end{document}
